% Created 2020-02-02 Κυρ 22:24
% Intended LaTeX compiler: pdflatex
\documentclass[11pt]{article}
\usepackage[utf8]{inputenc}
\usepackage[T1]{fontenc}
\usepackage{graphicx}
\usepackage{grffile}
\usepackage{longtable}
\usepackage{wrapfig}
\usepackage{rotating}
\usepackage[normalem]{ulem}
\usepackage{amsmath}
\usepackage{textcomp}
\usepackage{amssymb}
\usepackage{capt-of}
\usepackage{hyperref}
\usepackage[margin=0.5in]{geometry}
\author{Vagif Aliyev, Nikolas Vornehm, Sotiris Gkoulimaris}
\date{\today}
\title{bad bois}
\hypersetup{
 pdfauthor={Vagif Aliyev, Nikolas Vornehm, Sotiris Gkoulimaris},
 pdftitle={bad bois},
 pdfkeywords={},
 pdfsubject={},
 pdfcreator={Emacs 26.3 (Org mode 9.3)}, 
 pdflang={English}}
\begin{document}

\maketitle


\section{Introduction}
\label{sec:org63cfbbe}
For this assignment we were asked to implement algorithms that solve the linear
system \(A b = x\). The program itself used and expanded upon the Matrix.cpp and
Matrix.h libraries that were partially constructed in-class.

Our solution expands upon the Matrix class by defining the CSRMatrix, BandMatrix
and SymMatrix subclasses. The solvers we decided to implement are:

\begin{center}
\begin{tabular}{lllll}
 & Dense & CSR Sparse & Banded & Symmetric\\
\hline
Inverse & x &  &  & \\
Gaussian Elim & x &  &  & \\
LU Solver & x &  &  & \\
Conjugate Gradient & x & x & x & x\\
SOR & x & x & x & x\\
Chebyshev Iterative & x & x & x & x\\
Multi CG & x &  &  & \\
Multi SOR & x &  &  & \\
\end{tabular}
\end{center}


\section{Class Specifications}
\label{sec:orgcb9cdfc}

All classes have a similar class structure inhereted by the Matrix base class as
well as functions that are necessary to solve linear systems (such as
matVecMult).

\subsection{Matrix.cpp}
\label{sec:org44dcbfa}
Matrix.cpp is the base class of our library. It holds the private variables of
rows, cols and a 'values' unique pointer array that stores the values of our dense
matrix (surpriiiise!). A number of helper functions have been implemented to
facilitate the functionality of our class.
\subsubsection{Helper functions and Operations}
\label{sec:org4e0edcb}
Most of the implemented functions are self-explanatory. Note that appart from
printMatrix, every other function is limited to the base class.

matVecMult() and matMatMult() perform matrix-vector and matrix-matrix
multiplications. matVecMult() has a loop than runs N
times and a nested loop that runs N times as well; thus, the complexity of this
algorithm is \(O(N^2)\). matMatMult() has three loops that run N times, two of
which are nested; that results in a time complexity of \(O(N^3)\).

luDecomposition() performs the LU decomposition on our matrix. The algorithm
iterates through N rows, for the first row it makes 2N(N-1) operations, for the
second 2(N-1)(N-2) and so on; thus, the overall complexity of our algorithm is
of the order \(O(N^3)\).

generateSPD() is a helper function that generates Symmetric Possitive Definite
Matrices. It receives an integer as input (the type of SPD matrix to
be produced - check Matrix.h). For every matrix type, the algorithm loops over
all rows and columns using nested for loops; thus, the time complexity of our
algorithm is \(O(N^2)\).

\subsection{CSRMatrix.cpp}
\label{sec:org1e0138d}
CSRMatrix is derived from Matrix and corresponds to a sparse matrix stored in a
CSR format. CSR uses the values array to store the non-zero values as well as
two extra arrays to store row and column positions. Using this format we
have the advantage of saving up space and doing less operations when we compute
matrix-vector and matrix-matrix multiplications. One drawback would be that
iterating over the elements of the sparse matrix accsess all three different
arrays; memory jumps can be quite expensive and thus hindering performance.

\subsubsection{Helper functions and Operations}
\label{sec:org7b4e64c}
The self-explanatory functions are: dense2csr(), csr2dense(). 
compareMatResInner() and compareMatResOut() are used in tandem with the sort algorithm in matMatMult.

matVecMult() is similar to the base class. The main difference is that instead
of iterating over all rows and columns, we take advantage of the CSR format and
iterate over row position, using those indexes to access the column elements.
Thus, the time complexity of this algorithm is \(O(N^2)\)

In order to compute matMatMult() we decided to compute the individual multiplication
results, together with row and column indexes in a vector of matRes structs.
To elaborate, instead of adding the values during the for loop, we store them at
each iteration and end up with a list of all non zero values with their position
indexes (on the output matrix). To add them together, first sort them according
to column (compareMatResInner()) and then row (compareMatResOut()). Then, we
iterate through the sorted matRes elements and populate the output matrix's
values, row position and column index). The time complexity of this algorithm \(O(N^3)\).

\subsection{BandMatrix.cpp}
\label{sec:org717bbae}
BandMatrix is the banded form of a dense matrix. BandMatrix stores only the elements
in the bands of the dense matrix. Essentially, from an NxN dense matrix, we get
a NxM matrix, where M is the number of bands. The structure of the class is
changed accordingly. With this format, we can greatly decrease the ammount of 
memory required to store a given matrix and also speed up our computations, 
since we will only iterate over the bands instead of all the columns.

\subsubsection{Helper functions and Operations}
\label{sec:org1cfd843}
As with our Matrix and CSR class, a number of helper and operation functions have been
implemented. We defined the functions dense2band() and band2dense() to transform
a matrix from dense to banded form and vice-versa.

The function matVecMult() performs the same operation as the base function. This
time however, the computation is much faster, since instead of iterating over
all the columns, we only need to iterate through bands. As a result, we can have
significant speed-up since the time complexity of this function is now O(MN).

\subsection{Symmetric.cpp}
\label{sec:org961a573}
A symmetric matrix is a square matrix that is equal to its transpose, hence
halving the storage required.

\subsubsection{Helper functions and Operations}
\label{sec:org4bd4ef5}
As usual,  symm2band() and symm2dense() are used to transform a matrix from
dense to banded form and vice-versa. A MatVecMult has also been implemented for
use by the Solvers.


\section{Linear Solvers}
\label{sec:org1dcd4d1}
As mentioned in the introduction, a number of linear solvers have been
implemented, all of which are used for a defualt dense matrix.
\subsection{Inverse Solver}
\label{sec:orgcbcf56e}
This is the most basic solver we implemented. We use the inverse function
to compute the inverse of input matrix and then we perform a matVecMult
operation between the inverse and the input vector. This algorithm uses the
determinant(), coFactor() and adjugate() helper functions and as a result has
a pretty high time complexity \(O(N^4)\).

\subsection{Gaussian Elimination}
\label{sec:orgd45e084}
This is also a fairly basic solver. The algorithm we used is essentially the
one we implemented in ACSE-3. It computes the upper triangular matrix of our
input matrix and performs back substitution to find the solution to our linear
system. The complexity of this algorithm will depend on the individual
complexities of the our two functions (since they are called sequentially).
The upper triangle function has a time complexity of \(O(N^3)\). back substitution has a time
complexity of \(O(N^2)\). As a result, the overall time complexity of our Gauss Elimination is \(O(N^3)\).

\subsection{LU Solver}
\label{sec:org972a800}
The LU solver is also the same algorithm that we were taught in ACSE-3. It
calculates the Lower and Upper decompositions of our input matrix, then it
performs a forward substitution between the Lower matrix and the input vector,
followed by a backward substitution giving the result to the linear system. The time
complexity for the LU decomposition is O(N\textsuperscript{3}) the forward and
back substitutions have a time complexity of \(O(N^2)\). Thus, the overall time
complexity of this solver is \(O(N^3)\).

\subsection{Conjugate Gradient}
\label{sec:orge118032}
The Conjugate Gradient method is here implemented as an iterative solver.
The algorithm used here is found on \url{https://en.wikipedia.org/wiki/Conjugate\_gradient\_method}.
It is capable of computing for all the matrix types implemented in this library. 
The conjugate gradient takes adjantage of the Symmetric Positive Definite Matrices
by exploiting its properties.

\subsection{SOR}
\label{sec:org0bbda72}
The Successive Over-Relaxation (SOR) method is an iterative solver that make use
of an relaxation factor omega given by the user. Omega can take values between 
0 and 2 (0 < omega 2). When omega is set to 1 the SOR is equivalent to the 
Gauss-Seidel (and Jacobi). The algorithm used in this library is extracted from:
\url{https://en.wikipedia.org/wiki/Successive\_over-relaxation}.

\subsection{Chebyshev}
\label{sec:org60e7a9b}
Lastly, implementented a version of Chebyshev found in
\url{https://en.wikipedia.org/wiki/Chebyshev\_iteration}. This iterative solver takes
advantage of extra information (namely the upper and lower estimate of
eigenvalues) provided by the user to avoid computation of inner products (as
they can be quite expensive). Our algoithm's performance depends on the number
of iterations, the input information given by the user and the performane of
matVecMult. Assuming that the algorithm converges given the upper limit for
iterations, the time complexity of Chebyshev is identical to matVecMult.

\subsection{Multi Linear Solvers}
\label{sec:org574ff54}
Versions of SOR and CG were implemented to solve multiple vectors in matrix
form. Essentially we solve multile linear systems were the b vectors are stacked
as a matrix.
\end{document}
